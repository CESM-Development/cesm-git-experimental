%mct Users Guide
% R. L. Jacob
% J. W. Larson / MCS, Argonne National Laboratory
% First Version Begun 8/28/00
\documentclass{article}
\usepackage{epsfig}
\usepackage{graphicx}
%\usepackage{fancyheadings}

% Keep these dimensions

\textheight     9in \topmargin      0pt \headsep        22pt
\headheight     0pt

\textwidth      6in \oddsidemargin  0in \evensidemargin 0in

\marginparpush  0pt \pagestyle{plain}

\setlength{\hoffset}{0.25in}

% Headings
% --------
\pagestyle{plain}  % AFTER redefining \textheight etc.

%  \lhead[]{{\em NGC Design Document}}   % left part of header
%  \chead[]{}                  % center part of header
%  \rhead[]{\em {\today}}  % right part of header

 % \cfoot{\roman{page}}
  %\lfoot[]{}      % left part of footer
 % \rfoot[]{}      % right part of footer
 % \headrulewidth 0pt      % if you don't want a rule under the header
 % \footrulewidth 0pt      % if you don't want a rule above the footer

%......................................................................
%.............begin document.............

\begin{document}

\begin{sloppypar}
{\huge\bf
%%%
%%% Enter your title below (after deleting mine)
%%%
The Model Coupling Toolkit (mct) User's Guide
\\ }                     %%% IMPORTANT: Keep this \\ before the }
\end{sloppypar}

%%%
%%% Author names and affiliations go below, follow example
%%%
\vspace{.3in}
             R.~L.~Jacob\\
             J.~W.~Larson\\
             E.~Ong\\
\vspace{.2in} {\em Mathematics and Computer Science Division,
Argonne National Laboratory\\}

\vfill

%%%
%%% These lines are standard - keep them!
%%% Edit the ``has not been published'' as appropriated.
{\em This paper has not been published and should  be regarded as
an Internal Report from MCS. Permission to quote from this
Technical Note should be  obtained from the MCS Division of
Argonne National Laboratory.}

\vspace{0.4in}


\thispagestyle{empty}
\newpage

%.......................... END FIRST PAGE ......................

\pagenumbering{roman}

%......................... REVISION HISTORY ..........................

\newpage
\setcounter{page}{2}     %%%% Revision History starts at page ii

\addcontentsline{toc}{part}{Revision History}

\vspace*{\fill}

\centerline{\huge\bf Revision History}

\bigskip
\noindent{This Technical Note was produced for the Accelerated
Climate Prediction Initiative (ACPI) Avant Garde Project.}

\begin{center}
\begin{tabular}{|l|l|l|l|}\hline
{\bf Version} & {\bf Version} & {\bf Pages Affected/}   & {\bf Aproval}\\
{\bf Number}  & {\bf Date}    & {\bf Extent of Changes} & {\bf Authority}\\
\hline\hline Version 1$\beta$ & September 13, 2001      & First
draft (before review) &\\\hline
\end{tabular}
\end{center}

\vspace*{\fill}


%..........................  ABSTRACT ..................................
\newpage
\setcounter{page}{3}     %%%% abstract starts at page iii
\addcontentsline{toc}{part}{Abstract}

\vspace*{\fill}

\begin{abstract}
This document is describes the basic concepts behind the Model
Coupling Tookit, a Fortran 90 library for the construction of
distributed memory parallel earth system model couplers.   Examples
of usage in a unit testor are also described.
\end{abstract}

\vspace*{\fill}
\newpage

\tableofcontents
\newpage

% Switch page numbering to arabic numerals

\pagenumbering{arabic}

\part{What is the Model Coupling Toolkit?}

\section{Introduction}

The goal of the Accelerated Climate Prediction Initiative (ACPI)
Avant Garde Project was to construct a performance-portable
version of the Community Climate System Model (CCSM).  This new
model was to be built using solid software design and engineering
practices.  The CCSM marks the merger of the National Center for
Atmospheric Research (NCAR) Climate System Model (CSM) with the
Parallel Climate Model (PCM).   

One of the chief design
challenges for the CCSM was converting the coupler from a single
memory image design with shared memory parallellism to a distibuted memory 
image design.  The Model Coupling Toolkit was created to provide
a parallel coupling kernel which handles the basic tasks of transfering
data between the various grids of the component models.  The full
coupler of CCSM is built using MCT.

%\begin{figure}
%\epsfxsize=6.0in
%\centerline{\epsfbox{coupler-layers.eps} }
%\caption{Software layers for the coupler.}
%\label{fig:coupler-layers}
%\end{figure}

\section{Lowlevel Classes and Methods}
The lowest level of the toolkit is the definition of the internal 
data representation of the coupler (the {\tt AttrVect}), two simple 
schemes for one-and-two dimensional data decompositions (the 
{\tt GlobalMap} and {\tt GlobalSegMap} respectively), and indexing 
of local data ({\tt the Navigator}).  In addition to these basic 
classes, some simple, collective communications methods are defined
for the {\tt AttrVect} component.

\section{Higer level Classes and Methods}
This layer of the toolkit contains classes and methods built upon 
level 3  classes.  These include:  a time averaging component (the 
{\tt Accumulator}) and some simple collective communications 
methods associated with it; representation of sparse matrices 
(the {\tt SparseMatrix}) and matrix-{\tt AttrVect} operations;
a general definition of coordinate grids (the {\tt GeneralGrid});
a general representation of collective data transfer---whether it be 
a transpose or a parallel transfer between pools of processors that 
can be either on the same set of processors or differing sets of 
processors (the {\tt Router}) and communications methods that employ
it; a facility for merging flux and state data from multiple component 
models (design still under discussion); a highest-level, all-encompassing
 class that describes the exchange of data between two component 
models---the {\tt Contract} (design still in progress).


%\addcontentsline{toc}{part}{References}

%\bibliographystyle{apalike}   % for BibTeX - uses [Name, year] method??
 
%\bibliography{coupler}
\end{document}

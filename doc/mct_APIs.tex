%mct API Specification
% J.W. Larson / MCS, Argonne National Laboratory
% First Version Begun 8/28/00
\documentclass{article}
\usepackage{epsfig}
\usepackage{graphicx}
%\usepackage{fancyheadings}

% Keep these dimensions

\textheight     9in \topmargin      0pt \headsep        22pt
\headheight     0pt

\textwidth      6in \oddsidemargin  0in \evensidemargin 0in

\marginparpush  0pt \pagestyle{plain}

\setlength{\hoffset}{0.25in}

% Headings
% --------
\pagestyle{plain}  % AFTER redefining \textheight etc.

%  \lhead[]{{\em NGC Design Document}}   % left part of header
%  \chead[]{}                  % center part of header
%  \rhead[]{\em {\today}}  % right part of header

 % \cfoot{\roman{page}}
  %\lfoot[]{}      % left part of footer
 % \rfoot[]{}      % right part of footer
 % \headrulewidth 0pt      % if you don't want a rule under the header
 % \footrulewidth 0pt      % if you don't want a rule above the footer

%......................................................................
%.............begin document.............

\begin{document}

\begin{sloppypar}
{\huge\bf
%%%
%%% Enter your title below (after deleting mine)
%%%
The Model Coupling Toolkit (mct) API Definition Document
\\ }                     %%% IMPORTANT: Keep this \\ before the }
\end{sloppypar}

%%%
%%% Author names and affiliations go below, follow example
%%%
\vspace{.3in}
             J.~W.~Larson\\
             R.~L.~Jacob\\
             A.~N.~D.~Others\\
\vspace{.2in} {\em Mathematics and Computer Science Division,
Argonne National Laboratory\\}

\vfill

%%%
%%% These lines are standard - keep them!
%%% Edit the ``has not been published'' as appropriated.
{\em This paper has not been published and should  be regarded as
an Internal Report from MCS. Permission to quote from this
Technical Note should be  obtained from the MCS Division of
Argonne National Laboratory.}

\vspace{0.4in}


\thispagestyle{empty}
\newpage

%.......................... END FIRST PAGE ......................

\pagenumbering{roman}

%......................... REVISION HISTORY ..........................

\newpage
\setcounter{page}{2}     %%%% Revision History starts at page ii

\addcontentsline{toc}{part}{Revision History}

\vspace*{\fill}

\centerline{\huge\bf Revision History}

\bigskip
\noindent{This Technical Note was produced for the Accelerated
Climate Prediction Initiative (ACPI) Avant Garde Project.}

\begin{center}
\begin{tabular}{|l|l|l|l|}\hline
{\bf Version} & {\bf Version} & {\bf Pages Affected/}   & {\bf Aproval}\\
{\bf Number}  & {\bf Date}    & {\bf Extent of Changes} & {\bf Authority}\\
\hline\hline Version 1$\beta$ & December 13, 2000      & First
draft (before review) &\\\hline
\end{tabular}
\end{center}

\vspace*{\fill}


%..........................  ABSTRACT ..................................
\newpage
\setcounter{page}{3}     %%%% abstract starts at page iii
\addcontentsline{toc}{part}{Abstract}

\vspace*{\fill}

\begin{abstract}
In this document we state the Application Program Interfaces (APIs)
 for the model coupling toolkit (mct), which has been proposed as
a set of components to implement the Next-Generation
Coupler (NGC) for the Community Climate System Model (CCSM).
Analysis of the Coupler's requirements lead us to a layered
design capable of supporting the coupler's high-level
command/control functions and its relatively low-level grid
transformation and physics calculation functions.  This design
and the concept of a Model Coupling Toolkit (MCT) were introduced 
in the NGC proposed design document.
\end{abstract}

\vspace*{\fill}
\newpage

\tableofcontents
\newpage

% Switch page numbering to arabic numerals

\pagenumbering{arabic}

\part{What is the Model Coupling Toolkit?}

\section{Introduction}

The goal of the Accelerated Climate Prediction Initiative (ACPI)
Avant Garde Project is to construct a performance-portable
version of the Community Climate System Model (CCSM).  This new
model is to be built using solid software design and engineering
practices.  The CCSM marks the merger of the National Center for
Atmospheric Research (NCAR) Climate System Model (CSM) with the
Parallel Climate Model (PCM).   One of the chief design
challenges for the CCSM is the flux coupler, which has been
nicknamed {\em the Next-Generation Coupler} (NGC).

The NGC is expected to increase dramatically the flexibility of
the CCSM, and expand correspondingly the variety of research
possible using this model.  The NGC will be constructed using 
components in the mct.

In the design document, we outlined a layered design for the 
NGC (Figure \ref{fig:coupler-layers}).  Here we state in detail 
the API's for layers three and four of this design, which comprise 
the mct.

\begin{figure}
\epsfxsize=6.0in
\centerline{\epsfbox{coupler-layers.eps} }
\caption{Software layers for the coupler.}
\label{fig:coupler-layers}
\end{figure}

\section{Level 3 Classes and Methods}
The lowest level of the toolkit is the definition of the internal 
data representation of the coupler (the {\tt AttrVect}), two simple 
schemes for one-and-two dimensional data decompositions (the 
{\tt GlobalMap} and {\tt GlobalSegMap} respectively), and indexing 
of local data ({\tt the Navigator}).  In addition to these basic 
classes, some simple, collective communications methods are defined
for the {\tt AttrVect} component.

\section{Level Classes and Methods 4}
This layer of the toolkit contains classes and methods built upon 
level 3  classes.  These include:  a time averaging component (the 
{\tt Accumulator}) and some simple collective communications 
methods associated with it; representation of sparse matrices 
(the {\tt SparseMatrix}) and matrix-{\tt AttrVect} operations;
a general definition of coordinate grids (the {\tt GeneralGrid});
a general representation of collective data transfer---whether it be 
a transpose or a parallel transfer between pools of processors that 
can be either on the same set of processors or differing sets of 
processors (the {\tt Router}) and communications methods that employ
it; a facility for merging flux and state data from multiple component 
models (design still under discussion); a highest-level, all-encompassing
 class that describes the exchange of data between two component 
models---the {\tt Contract} (design still in progress).


\part{Level 3 API's}
\section{Documentation of the AttrVect Module}
%                **** IMPORTANT NOTICE *****
% This LaTeX file has been automatically produced by ProTeX v. 1.0
% Any changes made to this file will likely be lost next time
% this file is regenerated from its Fortran source.
% Send questions to Arlindo da Silva, dasilva@gsfc.nasa.gov
 
\parskip        0pt
\parindent      0pt
\baselineskip  11pt
 
%--------------------- SHORT-HAND MACROS ----------------------
\def\bv{\begin{verbatim}}
\def\ev{\end{verbatim}}
\def\be{\begin{equation}}
\def\ee{\end{equation}}
\def\bea{\begin{eqnarray}}
\def\eea{\end{eqnarray}}
\def\bi{\begin{itemize}}
\def\ei{\end{itemize}}
\def\bn{\begin{enumerate}}
\def\en{\end{enumerate}}
\def\bd{\begin{description}}
\def\ed{\end{description}}
\def\({\left (}
\def\){\right )}
\def\[{\left [}
\def\]{\right ]}
\def\<{\left  \langle}
\def\>{\right \rangle}
\def\cI{{\cal I}}
\def\diag{\mathop{\rm diag}}
\def\tr{\mathop{\rm tr}}
%-------------------------------------------------------------
 
%/////////////////////////////////////////////////////////////

 \subsection{Module m\_AttrVect - a distributed Innovation vector}


 
 
  An {\em attribute vector} is a scheme for storing bundles of integer 
  and real data vectors, indexed by lists of their respective attributes.
  The attribute vector is implemented in Fortran 90 using the 
  {\tt AttrVect} derived type.  This module contains the definition of
  {\tt AttrVect} class, and numerous methods that service it.
 
\bigskip{\sf INTERFACE:}
\begin{verbatim} 
 module m_AttrVect\end{verbatim}{\em USES:}
\begin{verbatim}      use m_List, only : List     Support for rList and iList components.
 
      implicit none
 
      private	  except
 
      public :: AttrVect          The class data structure
 
      public :: init		  create a local vector
      public :: clean		  clean the local vector
      public :: lsize		  size of the local vector
      public :: nIAttr		  number of integer attributes on local
      public :: nRAttr		  number of real attributes on local
      public :: indexIA		  index the integer attributes
      public :: indexRA		  index the real attributes
      public :: getIList          return list of integer attributes
      public :: getRList          return list of real attributes
      public :: Sort              sort entries, and return permutation
      public :: Permute           permute entries
      public :: SortPermute       sort and permute entries
 
    type AttrVect
      type(List) :: iList
      type(List) :: rList
      integer,dimension(:,:),pointer :: iAttr
      real   ,dimension(:,:),pointer :: rAttr
    end type AttrVect
 
    interface init   ; module procedure	&
 init_,	&
 initv_
    end interface
    interface clean  ; module procedure clean_  ; end interface
    interface lsize  ; module procedure lsize_  ; end interface
    interface nIAttr ; module procedure nIAttr_ ; end interface
    interface nRAttr ; module procedure nRAttr_ ; end interface
    interface indexIA; module procedure indexIA_; end interface
    interface indexRA; module procedure indexRA_; end interface
    interface getIList; module procedure getIList_; end interface
    interface getRList; module procedure getRList_; end interface
    interface Sort    ; module procedure Sort_    ; end interface
    interface Permute ; module procedure Permute_ ; end interface
    interface SortPermute ; module procedure SortPermute_ ; end interface
 \end{verbatim}{\sf REVISION HISTORY:}
\begin{verbatim}  	10Apr98 - Jing Guo <guo@thunder> - initial prototype/prolog/code
  	10Oct00 - J.W. Larson <larson@mcs.anl.gov> - made getIList
                  and getRList functions public and added appropriate
                  interface definitions
        20Oct00 - J.W. Larson <larson@mcs.anl.gov> - added Sort, 
                  Permute, and SortPermute functions.\end{verbatim}
 
%/////////////////////////////////////////////////////////////
 
\mbox{}\hrulefill\ 
 

 \subsubsection{init\_ - initialize with given iList, rList, and the size}


 
 
\bigskip{\sf INTERFACE:}
\begin{verbatim} 
 subroutine init_(aV,iList,rList,lsize)\end{verbatim}{\em USES:}
\begin{verbatim}      use m_List, only : init,nitem
      use m_mall
      use m_die
      implicit none
      type(AttrVect),intent(out) :: aV
      character(len=*),optional,intent(in) :: iList
      character(len=*),optional,intent(in) :: rList
      integer,         optional,intent(in) :: lsize
 \end{verbatim}{\sf REVISION HISTORY:}
\begin{verbatim}  	09Apr98 - Jing Guo <guo@thunder> - initial prototype/prolog/code\end{verbatim}
 
%/////////////////////////////////////////////////////////////
 
\mbox{}\hrulefill\ 
 

 \subsubsection{initv\_ - initialize on the vectors}


 
 
\bigskip{\sf INTERFACE:}
\begin{verbatim} 
 subroutine initv_(aV,bV,lsize)\end{verbatim}{\em USES:}
\begin{verbatim}      use m_String, only : String,char
      use m_List,   only : get
 
      implicit none
 
      type(AttrVect),intent(out) :: aV
      type(AttrVect),intent(in)  :: bV
      integer,       intent(in)  :: lsize
 \end{verbatim}{\sf REVISION HISTORY:}
\begin{verbatim}  	22Apr98 - Jing Guo <guo@thunder> - initial prototype/prolog/code\end{verbatim}
 
%/////////////////////////////////////////////////////////////
 
\mbox{}\hrulefill\ 
 

 \subsubsection{clean\_ - clean a vector}


 
 
\bigskip{\sf INTERFACE:}
\begin{verbatim} 
 subroutine clean_(aV)\end{verbatim}{\em USES:}
\begin{verbatim}      use m_mall
      use m_stdio
      use m_die
      use m_List, only : clean
 
      implicit none
 
      type(AttrVect),intent(inout) :: aV
 \end{verbatim}{\sf REVISION HISTORY:}
\begin{verbatim}  	09Apr98 - Jing Guo <guo@thunder> - initial prototype/prolog/code\end{verbatim}
 
%/////////////////////////////////////////////////////////////
 
\mbox{}\hrulefill\ 
 

 \subsubsection{lsize\_ - the local size of the vector}


 
 
\bigskip{\sf INTERFACE:}
\begin{verbatim} 
 function lsize_(aV)
 
     implicit none
 
      type(AttrVect), intent(in) :: aV
      integer :: lsize_
 \end{verbatim}{\sf REVISION HISTORY:}
\begin{verbatim}  	09Apr98 - Jing Guo <guo@thunder> - initial prototype/prolog/code\end{verbatim}
 
%/////////////////////////////////////////////////////////////
 
\mbox{}\hrulefill\ 
 

 \subsubsection{nIAttr\_ - number of INTEGER type attributes}


 
 
\bigskip{\sf INTERFACE:}
\begin{verbatim} 
 function nIAttr_(aV)\end{verbatim}{\em USES:}
\begin{verbatim}      use m_List, only : nitem
 
      implicit none
      type(AttrVect),intent(in) :: aV
      integer :: nIAttr_
 \end{verbatim}{\sf REVISION HISTORY:}
\begin{verbatim}  	22Apr98 - Jing Guo <guo@thunder> - initial prototype/prolog/code\end{verbatim}
 
%/////////////////////////////////////////////////////////////
 
\mbox{}\hrulefill\ 
 

 \subsubsection{nRAttr\_ - number of REAL type attributes}


 
 
\bigskip{\sf INTERFACE:}
\begin{verbatim} 
 function nRAttr_(aV)\end{verbatim}{\em USES:}
\begin{verbatim}      use m_List, only : nitem
 
      implicit none
 
      type(AttrVect),intent(in) :: aV
      integer :: nRAttr_
 \end{verbatim}{\sf REVISION HISTORY:}
\begin{verbatim}  	22Apr98 - Jing Guo <guo@thunder> - initial prototype/prolog/code\end{verbatim}
 
%/////////////////////////////////////////////////////////////
 
\mbox{}\hrulefill\ 
 

 \subsubsection{getIList\_ - get an item from iList}


 
 
\bigskip{\sf INTERFACE:}
\begin{verbatim} 
 subroutine getIList_(item,ith,aVect)\end{verbatim}{\em USES:}
\begin{verbatim}      use m_String, only : String
      use m_List,   only : get
 
      implicit none
 
      type(String),intent(out) :: item
      integer,     intent(in)  :: ith
      type(AttrVect),intent(in) :: aVect
 \end{verbatim}{\sf REVISION HISTORY:}
\begin{verbatim}  	24Apr98 - Jing Guo <guo@thunder> - initial prototype/prolog/code\end{verbatim}
 
%/////////////////////////////////////////////////////////////
 
\mbox{}\hrulefill\ 
 

 \subsubsection{getRList\_ - get an item from rList}


 
 
\bigskip{\sf INTERFACE:}
\begin{verbatim} 
 subroutine getRList_(item,ith,aVect)\end{verbatim}{\em USES:}
\begin{verbatim}      use m_String, only : String
      use m_List,   only : get
 
      implicit none
 
      type(String),intent(out) :: item
      integer,     intent(in)  :: ith
      type(AttrVect),intent(in) :: aVect
 \end{verbatim}{\sf REVISION HISTORY:}
\begin{verbatim}  	24Apr98 - Jing Guo <guo@thunder> - initial prototype/prolog/code\end{verbatim}
 
%/////////////////////////////////////////////////////////////
 
\mbox{}\hrulefill\ 
 

 \subsubsection{indexIA\_ - index the integer attribute List}


 
 
\bigskip{\sf INTERFACE:}
\begin{verbatim} 
 function indexIA_(aV,item,perrWith,dieWith)\end{verbatim}{\em USES:}
\begin{verbatim}      use m_List, only : index
      use m_die,  only : die
      use m_stdio,only : stderr
 
      implicit none
 
      type(AttrVect), intent(in) :: aV
      character(len=*),intent(in) :: item
      character(len=*),optional,intent(in) :: perrWith
      character(len=*),optional,intent(in) :: dieWith
      integer :: indexIA_
 \end{verbatim}{\sf REVISION HISTORY:}
\begin{verbatim}  	27Apr98 - Jing Guo <guo@thunder> - initial prototype/prolog/code\end{verbatim}
 
%/////////////////////////////////////////////////////////////
 
\mbox{}\hrulefill\ 
 

 \subsubsection{indexRA\_ - index the integer attribute List}


 
  
 
\bigskip{\sf INTERFACE:}
\begin{verbatim} 
 function indexRA_(aV,item,perrWith,dieWith)\end{verbatim}{\em USES:}
\begin{verbatim}      use m_List, only : index
      use m_die,  only : die
      use m_stdio,only : stderr
 
      implicit none
 
      type(AttrVect), intent(in) :: aV
      character(len=*),intent(in) :: item
      character(len=*),optional,intent(in) :: perrWith
      character(len=*),optional,intent(in) :: dieWith
      integer :: indexRA_
 \end{verbatim}{\sf REVISION HISTORY:}
\begin{verbatim}  	27Apr98 - Jing Guo <guo@thunder> - initial prototype/prolog/code\end{verbatim}
 
%/////////////////////////////////////////////////////////////
 
\mbox{}\hrulefill\ 
 

 \subsubsection{Sort\_ - return index permutation keyed by a list of}


             attributes
 
  The subroutine {\tt Sort\_()} uses a list of keys defined by the {\tt List} 
  {\tt sList}, searches for the appropriate integer or real attributes
  referenced by the items in {\tt sList} ( that is, it identifies the 
  appropriate entries in {aV\%iList} and {\tt aV\%rList}), and then 
  uses these keys to generate a permutation {\tt perm} that will put
  the entries of the attribute vector {\tt aV} in lexicographic order
  as defined by {\tt sList} (the ordering in {\tt sList} being from
  left to right.
 
  {\bf N.B.:}  This routine will fail if {\tt aV\%rList} and 
  {\tt aV\%rList} share one or more common entries. 
 
  {\bf N.B.:}  This routine will fail if {\tt aV\%rList} and 
 
\bigskip{\sf INTERFACE:}
\begin{verbatim} 
 subroutine Sort_(aV, sList, perm, descend, perrWith, dieWith)\end{verbatim}{\em USES:}
\begin{verbatim}      use m_String,        only : String
      use m_String,        only : String_tochar => tochar
      use m_List ,         only : List_index => index
      use m_List ,         only : List_nitem => nitem
      use m_List ,         only : List_get   => get
      use m_die ,          only : die
      use m_stdio ,        only : stderr
      use m_SortingTools , only : IndexSet
      use m_SortingTools , only : IndexSort
 
      implicit none
 
      type(AttrVect), intent(in)                  :: aV
      type(List),     intent(in)                  :: sList
      integer, dimension(:), pointer              :: perm
      logical, dimension(:), optional, intent(in) :: descend
      character(len=*), optional, intent(in)      :: perrWith
      character(len=*), optional, intent(in)      :: dieWith
 
 \end{verbatim}{\sf REVISION HISTORY:}
\begin{verbatim}  	20Oct00 - J.W. Larson <larson@mcs.anl.gov> - initial prototype\end{verbatim}
 
%/////////////////////////////////////////////////////////////
 
\mbox{}\hrulefill\ 
 

 \subsubsection{Permute\_ - return index permutation keyed by a list of}


             attributes
 
  The subroutine {\tt Permute\_()} uses a a permutation {\tt perm} (which can
  be generated by the routine {\tt Sort\_()} in this module) to rearrange
  the entries in the attribute integer and real storage areas of the
  input attribute vector {\tt aV}--{\tt aV\%iAttr} and {\tt aV\%rAttr}, 
  respectively.
 
\bigskip{\sf INTERFACE:}
\begin{verbatim} 
 subroutine Permute_(aV, perm, perrWith, dieWith)\end{verbatim}{\em USES:}
\begin{verbatim}      use m_die ,          only : die
      use m_stdio ,        only : stderr
      use m_SortingTools , only : Permute
 
      implicit none
 
      type(AttrVect), intent(inout) :: aV
      integer, dimension(:), intent(in) :: perm
      character(len=*),optional,intent(in) :: perrWith
      character(len=*),optional,intent(in) :: dieWith
 \end{verbatim}{\sf REVISION HISTORY:}
\begin{verbatim}  	23Oct00 - J.W. Larson <larson@mcs.anl.gov> - initial prototype\end{verbatim}
 
%/////////////////////////////////////////////////////////////
 
\mbox{}\hrulefill\ 
 

 \subsubsection{SortPermute\_ - Place AttrVect data in lexicographic order.}


 
 
  The subroutine {\tt SortPermute\_()} uses the routine {\tt Sort\_()} 
  to create an index permutation {\tt perm} that will place the AttrVect
  entries in the lexicographic order defined by the keys in the List 
  variable {\tt key\_list}.  This permutation is then used by the routine
  {\tt Permute\_()} to place the AttreVect entries in lexicographic order.
 
\bigskip{\sf INTERFACE:}
\begin{verbatim} 
 subroutine SortPermute_(aV, key_list, descend, perrWith, dieWith)\end{verbatim}{\em USES:}
\begin{verbatim}      use m_die ,          only : die
      use m_stdio ,        only : stderr
 
      implicit none
 
      type(AttrVect), intent(inout) :: aV
      type(List), intent(in)        :: key_list
      logical , dimension(:), optional, intent(in) :: descend
      character(len=*), optional, intent(in) :: perrWith
      character(len=*), optional, intent(in) :: dieWith
 
 \end{verbatim}{\sf REVISION HISTORY:}
\begin{verbatim}  	24Oct00 - J.W. Larson <larson@mcs.anl.gov> - initial prototype\end{verbatim}

%...............................................................

\section{Documentation of the Navigator Module}
%                **** IMPORTANT NOTICE *****
% This LaTeX file has been automatically produced by ProTeX v. 1.0
% Any changes made to this file will likely be lost next time
% this file is regenerated from its Fortran source.
% Send questions to Arlindo da Silva, dasilva@gsfc.nasa.gov
 
\parskip        0pt
\parindent      0pt
\baselineskip  11pt
 
%--------------------- SHORT-HAND MACROS ----------------------
\def\bv{\begin{verbatim}}
\def\ev{\end{verbatim}}
\def\be{\begin{equation}}
\def\ee{\end{equation}}
\def\bea{\begin{eqnarray}}
\def\eea{\end{eqnarray}}
\def\bi{\begin{itemize}}
\def\ei{\end{itemize}}
\def\bn{\begin{enumerate}}
\def\en{\end{enumerate}}
\def\bd{\begin{description}}
\def\ed{\end{description}}
\def\({\left (}
\def\){\right )}
\def\[{\left [}
\def\]{\right ]}
\def\<{\left  \langle}
\def\>{\right \rangle}
\def\cI{{\cal I}}
\def\diag{\mathop{\rm diag}}
\def\tr{\mathop{\rm tr}}
%-------------------------------------------------------------
 
%/////////////////////////////////////////////////////////////

 \subsection{Module m\_Navigator - Array of pointers}


 
 
\bigskip{\sf INTERFACE:}
\begin{verbatim} 
    module m_Navigator
      implicit none
      private	  except
 
      public :: Navigator		  The class data structure
      public :: Navigator_init,init	  initialize an object
      public :: clean			  clean an object
      public :: lsize			  the true size
      public :: msize			  the maximum size
      public :: resize			  adjust the true size
      public :: get			  get an entry
 
      public :: ptr_displs		  referencing %displs(:)
      public :: ptr_counts		  referencing %counts(:)
 
    type Navigator
      integer :: lsize	  true size, if over-dimensioned
      integer,pointer,dimension(:) :: displs
      integer,pointer,dimension(:) :: counts
    end type Navigator
 
    interface Navigator_init; module procedure	&
 init_; end interface
    interface init  ; module procedure init_  ; end interface
    interface clean ; module procedure clean_ ; end interface
    interface lsize ; module procedure lsize_ ; end interface
    interface msize ; module procedure msize_ ; end interface
    interface resize; module procedure resize_; end interface
    interface get   ; module procedure get_   ; end interface
    interface ptr_displs; module procedure	&
 ptr_displs_; end interface
    interface ptr_counts; module procedure	&
 ptr_counts_; end interface
 \end{verbatim}{\sf REVISION HISTORY:}
\begin{verbatim}  	22May00	- Jing Guo <guo@dao.gsfc.nasa.gov>
 		- initial prototype/prolog/code
  	21Oct00	- J.W. Larson <larson@mcs.anl.gov>
                  minor modification (removal of 'private' 
                  statement in declaration of Navigator.  This
                  will be put back as the MCT matures.\end{verbatim}
 
%/////////////////////////////////////////////////////////////
 
\mbox{}\hrulefill\ 
 

 \subsubsection{init\_ - initialize a navigator}


 
 
\bigskip{\sf INTERFACE:}
\begin{verbatim} 
    subroutine init_(nav,lsize,stat)
      use m_mall,only : mall_ison,mall_mci
      use m_die ,only : die,perr
      implicit none
      type(Navigator),intent(out) :: nav	  the object
      integer,intent(in) :: lsize		  nominal size
      integer,optional,intent(out) :: stat	  status
 \end{verbatim}{\sf REVISION HISTORY:}
\begin{verbatim}  	22May00	- Jing Guo <guo@dao.gsfc.nasa.gov>
 		- initial prototype/prolog/code\end{verbatim}
 
%/////////////////////////////////////////////////////////////
 
\mbox{}\hrulefill\ 
 

 \subsubsection{clean\_ - clean a navigator}


 
 
\bigskip{\sf INTERFACE:}
\begin{verbatim} 
    subroutine clean_(nav,stat)
      use m_mall,only : mall_ison,mall_mco
      use m_die ,only : die,perr
      implicit none
      type(Navigator),intent(inout) :: nav	  the object
      integer,optional,intent(out) :: stat	  status
 \end{verbatim}{\sf REVISION HISTORY:}
\begin{verbatim}  	22May00	- Jing Guo <guo@dao.gsfc.nasa.gov>
 		- initial prototype/prolog/code\end{verbatim}
 
%/////////////////////////////////////////////////////////////
 
\mbox{}\hrulefill\ 
 

 \subsubsection{lsize\_ - return the true size}


 
 
\bigskip{\sf INTERFACE:}
\begin{verbatim} 
    function lsize_(nav)
      implicit none
      type(Navigator),intent(in) :: nav
      integer :: lsize_
 \end{verbatim}{\sf REVISION HISTORY:}
\begin{verbatim}  	22May00	- Jing Guo <guo@dao.gsfc.nasa.gov>
 		- initial prototype/prolog/code\end{verbatim}
 
%/////////////////////////////////////////////////////////////
 
\mbox{}\hrulefill\ 
 

 \subsubsection{msize\_ - return the maximum size}


 
 
\bigskip{\sf INTERFACE:}
\begin{verbatim} 
    function msize_(nav)
      implicit none
      type(Navigator),intent(in) :: nav
      integer :: msize_
 \end{verbatim}{\sf REVISION HISTORY:}
\begin{verbatim}  	22May00	- Jing Guo <guo@dao.gsfc.nasa.gov>
 		- initial prototype/prolog/code\end{verbatim}
 
%/////////////////////////////////////////////////////////////
 
\mbox{}\hrulefill\ 
 

 \subsubsection{resize\_ - adjust the true size}


 
 
        It is user's responsibility to ensure lsize is no greater than
    the maximum size of the vector.
 
        If lsize is not specified, the size of the vector is adjusted
    to its original size.
 
 
\bigskip{\sf INTERFACE:}
\begin{verbatim} 
    subroutine resize_(nav,lsize)
      implicit none
      type(Navigator),intent(inout) :: nav
      integer,optional,intent(in) :: lsize
 \end{verbatim}{\sf REVISION HISTORY:}
\begin{verbatim}  	22May00	- Jing Guo <guo@dao.gsfc.nasa.gov>
 		- initial prototype/prolog/code\end{verbatim}
 
%/////////////////////////////////////////////////////////////
 
\mbox{}\hrulefill\ 
 

 \subsubsection{get\_ - get an entry according to the user's preference}


 
 
\bigskip{\sf INTERFACE:}
\begin{verbatim} 
    subroutine get_(nav,inav,displ,count,lc,ln,le)
      implicit none
      type(Navigator),intent(in) :: nav
      integer,intent(in) :: inav
      integer,optional,intent(out) :: displ
      integer,optional,intent(out) :: count
      integer,optional,intent(out) :: lc
      integer,optional,intent(out) :: ln
      integer,optional,intent(out) :: le
 \end{verbatim}{\sf REVISION HISTORY:}
\begin{verbatim}  	22May00	- Jing Guo <guo@dao.gsfc.nasa.gov>
 		- initial prototype/prolog/code\end{verbatim}
 
%/////////////////////////////////////////////////////////////
 
\mbox{}\hrulefill\ 
 

 \subsubsection{ptr\_displs\_ - returns pointer to displs(:) component.}


 
 
\bigskip{\sf INTERFACE:}
\begin{verbatim} 
    function ptr_displs_(nav,lbnd,ubnd)
      implicit none
      type(Navigator),intent(in) :: nav
      integer,optional,intent(in) :: lbnd
      integer,optional,intent(in) :: ubnd
      integer,pointer,dimension(:) :: ptr_displs_
 \end{verbatim}{\sf REVISION HISTORY:}
\begin{verbatim}  	22May00	- Jing Guo <guo@dao.gsfc.nasa.gov>
 		- initial prototype/prolog/code\end{verbatim}
 
%/////////////////////////////////////////////////////////////
 
\mbox{}\hrulefill\ 
 

 \subsubsection{ptr\_counts\_ - returns pointer to counts(:) component.}


 
 
\bigskip{\sf INTERFACE:}
\begin{verbatim} 
    function ptr_counts_(nav,lbnd,ubnd)
      implicit none
      type(Navigator),intent(in) :: nav
      integer,optional,intent(in) :: lbnd
      integer,optional,intent(in) :: ubnd
      integer,pointer,dimension(:) :: ptr_counts_
 \end{verbatim}{\sf REVISION HISTORY:}
\begin{verbatim}  	22May00	- Jing Guo <guo@dao.gsfc.nasa.gov>
 		- initial prototype/prolog/code\end{verbatim}

%...............................................................

\section{Documentation of the GlobalMap Module}
%                **** IMPORTANT NOTICE *****
% This LaTeX file has been automatically produced by ProTeX v. 1.0
% Any changes made to this file will likely be lost next time
% this file is regenerated from its Fortran source.
% Send questions to Arlindo da Silva, dasilva@gsfc.nasa.gov
 
\parskip        0pt
\parindent      0pt
\baselineskip  11pt
 
%--------------------- SHORT-HAND MACROS ----------------------
\def\bv{\begin{verbatim}}
\def\ev{\end{verbatim}}
\def\be{\begin{equation}}
\def\ee{\end{equation}}
\def\bea{\begin{eqnarray}}
\def\eea{\end{eqnarray}}
\def\bi{\begin{itemize}}
\def\ei{\end{itemize}}
\def\bn{\begin{enumerate}}
\def\en{\end{enumerate}}
\def\bd{\begin{description}}
\def\ed{\end{description}}
\def\({\left (}
\def\){\right )}
\def\[{\left [}
\def\]{\right ]}
\def\<{\left  \langle}
\def\>{\right \rangle}
\def\cI{{\cal I}}
\def\diag{\mathop{\rm diag}}
\def\tr{\mathop{\rm tr}}
%-------------------------------------------------------------
 
%/////////////////////////////////////////////////////////////

 \subsection{Module m\_GlobalMap - a size description of a distributed array}


 
 
\bigskip{\sf INTERFACE:}
\begin{verbatim} 
    module m_GlobalMap
      implicit none
      private	  except
 
      public :: GlobalMap		  The class data structure
      public :: gsize
      public :: lsize
      public :: init
      public :: init_remote
      public :: clean
      public :: rank
      public :: local_range
 
    type GlobalMap
      integer :: gsize				  the Global size
      integer :: lsize				  my local size
      integer :: lleft				  my left index
      integer,dimension(:),pointer :: counts	  all local sizes
      integer,dimension(:),pointer :: displs	  PE ordered locations
    end type GlobalMap
 
    interface gsize; module procedure gsize_; end interface
    interface lsize; module procedure lsize_; end interface
    interface init ; module procedure	&
 initd_,	&	  initialize from all PEs
 initr_		  initialize from the root
    end interface
    interface init_remote; module procedure init_remote_; end interface
    interface clean; module procedure clean_; end interface
    interface rank ; module procedure rank_ ; end interface
    interface local_range; module procedure range_; end interface
 \end{verbatim}{\sf REVISION HISTORY:}
\begin{verbatim}  	21Apr98 - Jing Guo <guo@thunder> - initial prototype/prolog/code
  	09Nov00 - J.W. Larson <larson@mcs.anl.gov> - added init_remote
                  interface.\end{verbatim}
 
%/////////////////////////////////////////////////////////////
 
\mbox{}\hrulefill\ 
 

 \subsubsection{initd\_ - define the map from distributed data}


 
 
\bigskip{\sf INTERFACE:}
\begin{verbatim} 
    subroutine initd_(GMap,ln,comm)
      use m_mpif90
      use m_die
      implicit none
      type(GlobalMap),intent(out) :: GMap
      integer,intent(in) :: ln	  the local size
      integer,intent(in) :: comm
 \end{verbatim}{\sf REVISION HISTORY:}
\begin{verbatim}  	21Apr98 - Jing Guo <guo@thunder> - initial prototype/prolog/code\end{verbatim}
 
%/////////////////////////////////////////////////////////////
 
\mbox{}\hrulefill\ 
 

 \subsubsection{initr\_ initialize the map from the root}


 
 
\bigskip{\sf INTERFACE:}
\begin{verbatim} 
    subroutine initr_(GMap,lns,root,comm)
      use m_mpif90
      use m_die
      use m_stdio
      implicit none
      type(GlobalMap),intent(out) :: GMap
      integer,dimension(:),intent(in) :: lns	  the distributed sizes
      integer,intent(in) :: root
      integer,intent(in) :: comm
 \end{verbatim}{\sf REVISION HISTORY:}
\begin{verbatim}  	29May98 - Jing Guo <guo@thunder> - initial prototype/prolog/code\end{verbatim}
 
%/////////////////////////////////////////////////////////////
 
\mbox{}\hrulefill\ 
 

 \subsubsection{init\_remote\_ initialize remote GlobalMap from the root}


 
  This method takes data describing a GlobalMap for another communicator
 
\bigskip{\sf INTERFACE:}
\begin{verbatim} 
 subroutine init_remote_(GMap, remote_lns, remote_npes, my_root, &
                     my_comm, remote_comm)
      use m_mpif90
      use m_die
      use m_stdio
      implicit none
      type(GlobalMap),intent(out) :: GMap
      integer, dimension(:) :: remote_lns       distributed sizes on 
                                                the remote communicator
      integer            :: remote_npes         number of pe's on remote 
                                                communicator
      integer,intent(in) :: my_root             my root
      integer,intent(in) :: my_comm             my communicator
      integer,intent(in) :: remote_comm         remote communicator
 \end{verbatim}{\sf REVISION HISTORY:}
\begin{verbatim}  	08Nov00 - J.W. Larson <larson@mcs.anl.gov> - initial prototype\end{verbatim}
 
%/////////////////////////////////////////////////////////////
 
\mbox{}\hrulefill\ 
 

 \subsubsection{clean\_ - clean the map}


 
 
\bigskip{\sf INTERFACE:}
\begin{verbatim} 
    subroutine clean_(GMap)
      use m_die
      implicit none
      type(GlobalMap),intent(inout) :: GMap
 \end{verbatim}{\sf REVISION HISTORY:}
\begin{verbatim}  	21Apr98 - Jing Guo <guo@thunder> - initial prototype/prolog/code\end{verbatim}
 
%/////////////////////////////////////////////////////////////
 
\mbox{}\hrulefill\ 
 

 \subsubsection{lsize\_ - find the local size from the map}


 
 
\bigskip{\sf INTERFACE:}
\begin{verbatim} 
    function lsize_(GMap)
      implicit none
      type(GlobalMap),intent(in) :: GMap
      integer :: lsize_
 \end{verbatim}{\sf REVISION HISTORY:}
\begin{verbatim}  	21Apr98 - Jing Guo <guo@thunder> - initial prototype/prolog/code\end{verbatim}
 
%/////////////////////////////////////////////////////////////
 
\mbox{}\hrulefill\ 
 

 \subsubsection{gsize\_ - find the global size from the map}


 
 
\bigskip{\sf INTERFACE:}
\begin{verbatim} 
    function gsize_(GMap)
      implicit none
      type(GlobalMap),intent(in) :: GMap
      integer :: gsize_
 \end{verbatim}{\sf REVISION HISTORY:}
\begin{verbatim}  	21Apr98 - Jing Guo <guo@thunder> - initial prototype/prolog/code\end{verbatim}
 
%/////////////////////////////////////////////////////////////
 
\mbox{}\hrulefill\ 
 

 \subsubsection{rank\_ - rank (which PE) of a given global index}


 
 
\bigskip{\sf INTERFACE:}
\begin{verbatim} 
    subroutine rank_(GMap,i_g,rank)
      implicit none
      type(GlobalMap),intent(in) :: GMap
      integer, intent(in)  :: i_g	  a global index
      integer, intent(out) :: rank
 \end{verbatim}{\sf REVISION HISTORY:}
\begin{verbatim}  	05May98 - Jing Guo <guo@thunder> - initial prototype/prolog/code\end{verbatim}
 
%/////////////////////////////////////////////////////////////
 
\mbox{}\hrulefill\ 
 

 \subsubsection{range\_ - the range of the local indices}


 
 
\bigskip{\sf INTERFACE:}
\begin{verbatim} 
    subroutine range_(GMap,lbnd,ubnd)
      implicit none
      type(GlobalMap),intent(in) :: GMap
      integer,intent(out) :: lbnd
      integer,intent(out) :: ubnd
 \end{verbatim}{\sf REVISION HISTORY:}
\begin{verbatim}  	05May98 - Jing Guo <guo@thunder> - initial prototype/prolog/code\end{verbatim}

%...............................................................

\section{Documentation of the GlobalSegMap Module}
%                **** IMPORTANT NOTICE *****
% This LaTeX file has been automatically produced by ProTeX v. 1.0
% Any changes made to this file will likely be lost next time
% this file is regenerated from its Fortran source.
% Send questions to Arlindo da Silva, dasilva@gsfc.nasa.gov
 
\parskip        0pt
\parindent      0pt
\baselineskip  11pt
 
%--------------------- SHORT-HAND MACROS ----------------------
\def\bv{\begin{verbatim}}
\def\ev{\end{verbatim}}
\def\be{\begin{equation}}
\def\ee{\end{equation}}
\def\bea{\begin{eqnarray}}
\def\eea{\end{eqnarray}}
\def\bi{\begin{itemize}}
\def\ei{\end{itemize}}
\def\bn{\begin{enumerate}}
\def\en{\end{enumerate}}
\def\bd{\begin{description}}
\def\ed{\end{description}}
\def\({\left (}
\def\){\right )}
\def\[{\left [}
\def\]{\right ]}
\def\<{\left  \langle}
\def\>{\right \rangle}
\def\cI{{\cal I}}
\def\diag{\mathop{\rm diag}}
\def\tr{\mathop{\rm tr}}
%-------------------------------------------------------------
 
%/////////////////////////////////////////////////////////////

 \subsection{Module m\_GlobalSegMap - a nontrivial 1-D decomposition of an array.}


 
  Consider the problem of the 1-dimensional decomposition of an array 
  across multiple processes.  If each process owns only one contiguous 
  segment, then the {\tt GlobalMap} (see {\tt m\_GlobalMap} or details) 
  is sufficient to describe the decomposition.  If, however, each  
  process owns multiple, non-adjacent segments of the array, a more 
  sophisticated approach is needed.   The {\tt GlobalSegMap} data type 
  allows one to describe a one-dimensional decomposition of an array
  with each process owning multiple, non-adjacent segments of the array.
 
  In the current implementation of the {\tt GlobalSegMap}, there is no
  santity check to guarantee that 
 $${\tt GlobalSegMap\%gsize} = \sum_{{\tt i}=1}^{\tt ngseg} 
  {\tt GlobalSegMap\%length(i)} . $$
  The reason we have not implemented such a check is to allow the user
  to use the {\tt GlobalSegMap} type to support decompositions of both 
  {\em haloed} and {\em masked} data.
 
\bigskip{\sf INTERFACE:}
\begin{verbatim} 
 module m_GlobalSegMap
 
      implicit none
 
      private	  except
 
      public :: GlobalSegMap	  The class data structure
      public :: init              Create
      public :: clean             Destroy
      public :: comm              Return communication
      public :: gsize             Return global vector size (excl. halos)
      public :: lsize             Return local storage size (incl. halos)
      public :: ngseg             Return global number of segments
      public :: nlseg             Return local number of segments
 
    type GlobalSegMap
      integer :: comm				  Communicator handle
      integer :: ngseg				  No. of Global segments
      integer :: gsize				  No. of Global elements
      integer :: lsize				  No. of Local elements
      integer,dimension(:),pointer :: start	  global seg. start index
      integer,dimension(:),pointer :: length	  segment lengths
      integer,dimension(:),pointer :: pe_loc	  PE locations
    end type GlobalSegMap
 
    interface init ; module procedure	&
 initd_,	&	  initialize from all PEs
 initr_		  initialize from the root
    end interface
    interface clean ; module procedure clean_ ; end interface
    interface comm  ; module procedure comm_  ; end interface
    interface gsize ; module procedure gsize_ ; end interface
    interface lsize ; module procedure lsize_ ; end interface
    interface ngseg ; module procedure ngseg_ ; end interface
    interface nlseg ; module procedure nlseg_ ; end interface
    interface rank  ; module procedure &
 rank1_ , &	  single rank case
 rankm_	          degenerate (multiple) ranks for halo case
    end interface
 \end{verbatim}{\sf REVISION HISTORY:}
\begin{verbatim}  	28Sep98 - J.W. Larson <larson@mcs.anl.gov> - initial prototype\end{verbatim}
 
%/////////////////////////////////////////////////////////////
 
\mbox{}\hrulefill\ 
 

 \subsubsection{initd\_ - define the map from distributed data}


 
 
\bigskip{\sf INTERFACE:}
\begin{verbatim} 
 subroutine initd_(GSMap, start, length, root, my_comm, &
                   gsm_comm, pe_loc, gsize)
 
 initd_(GSMap, comm, gsize,)\end{verbatim}{\em USES:}
\begin{verbatim}      use m_mpif90
      use m_die
      use m_stdio
 
      implicit none
 
      type(GlobalSegMap),intent(out)  :: GSMap     Output GlobalSegMap
 
      integer,dimension(:),intent(in) :: start     segment local start index 
      integer,dimension(:),intent(in) :: length    the distributed sizes
      integer,intent(in)              :: root      root on my_com
      integer,intent(in)              :: my_comm   local communicatior
      integer,intent(in), optional    :: gsm_comm   communicator for the
                                                    output GlobalSegMap
      integer,dimension(:), pointer, optional :: pe_loc   process location
      integer,intent(in), optional    :: gsize     global vector size
                                                   (optional).  It can
                                                   be computed by this 
                                                   routine if no haloing
                                                   is assumed.
 \end{verbatim}{\sf REVISION HISTORY:}
\begin{verbatim}  	29Sep98 - J.W. Larson <larson@mcs.anl.gov> - initial prototype
  	14Nov00 - J.W. Larson <larson@mcs.anl.gov> - final working version\end{verbatim}
 
%/////////////////////////////////////////////////////////////
 
\mbox{}\hrulefill\ 
 

 \subsubsection{initr\_ initialize the map from the root}


 
 
\bigskip{\sf INTERFACE:}
\begin{verbatim} 
 subroutine initr_(GSMap, ngseg, start, length, pe_loc, root,  &
                   my_comm, gsm_comm, gsize)
  !Uses:
      use m_mpif90
      use m_die
      use m_stdio
 
     implicit none
 
      type(GlobalSegMap),intent(out)  :: GSMap     Output GlobalSegMap
      integer, intent(in)             :: ngseg     no. of global segments
      integer,dimension(:),intent(in) :: start     segment local start index 
      integer,dimension(:),intent(in) :: length    the distributed sizes
      integer,dimension(:),intent(in) :: pe_loc    process location
      integer,intent(in)              :: root      root on my_com
      integer,intent(in)              :: my_comm   local communicatior
      integer,intent(in), optional    :: gsm_comm   communicator for the
                                                    output GlobalSegMap
      integer,intent(in), optional    :: gsize     global vector size
                                                   (optional).  It can
                                                   be computed by this 
                                                   routine if no haloing
                                                   is assumed.\end{verbatim}{\sf REVISION HISTORY:}
\begin{verbatim}  	29Sep98 - J.W. Larson <larson@mcs.anl.gov> - initial prototype
  	09Nov98 - J.W. Larson <larson@mcs.anl.gov> - final working version\end{verbatim}
 
%/////////////////////////////////////////////////////////////
 
\mbox{}\hrulefill\ 
 

 \subsubsection{clean\_ - clean the map}


 
  This routine deallocates the array components of the {\tt GlobalSegMap}
  argument {\tt GSMap}: {\tt GSMap\%start}, {\tt GSMap\%length}, and
  {\tt GSMap\%pe\_loc}.  It also zeroes out the values of the integer
  components {\tt GSMap\%ngseg}, {\tt GSMap\%comm}, {\tt GSMap\%gsize},
  and {\tt GSMap\%lsize}.
 
\bigskip{\sf INTERFACE:}
\begin{verbatim} 
    subroutine clean_(GSMap)\end{verbatim}{\em USES:}
\begin{verbatim}      use m_die
 
      implicit none
 
      type(GlobalSegMap),intent(inout) :: GSMap
 \end{verbatim}{\sf REVISION HISTORY:}
\begin{verbatim}  	29Sep98 - J.W. Larson <larson@mcs.anl.gov> - initial prototype\end{verbatim}
 
%/////////////////////////////////////////////////////////////
 
\mbox{}\hrulefill\ 
 

 \subsubsection{ngseg\_ - Return the global number of segments from the map}


 
  The function {\tt ngseg\_()} returns the global number of vector
  segments in the {\tt GlobalSegMap} argument {\tt GSMap}.  This is
  merely the value of {\tt GSMap\%ngseg}.
 
\bigskip{\sf INTERFACE:}
\begin{verbatim} 
 function ngseg_(GSMap)
 
      implicit none
 
      type(GlobalSegMap),intent(in) :: GSMap
      integer :: ngseg_
 \end{verbatim}{\sf REVISION HISTORY:}
\begin{verbatim}  	29Sep98 - J.W. Larson <larson@mcs.anl.gov> - initial prototype\end{verbatim}
 
%/////////////////////////////////////////////////////////////
 
\mbox{}\hrulefill\ 
 

 \subsubsection{nlseg\_ - Return the global number of segments from the map}


 
  The function {\tt nlseg\_()} returns the number of vector segments 
  in the {\tt GlobalSegMap} argument {\tt GSMap} that reside on the 
  process specified by the input argument {\tt pID}.  This is the 
  number of entries {\tt GSMap\%pe\_loc} whose value equals {\tt pID}.
 
\bigskip{\sf INTERFACE:}
\begin{verbatim} 
 function nlseg_(GSMap, pID)
 
      implicit none
 
      type(GlobalSegMap),intent(in) :: GSMap
      integer,           intent(in) :: pID
 
      integer :: nlseg_
 \end{verbatim}{\sf REVISION HISTORY:}
\begin{verbatim}  	29Sep98 - J.W. Larson <larson@mcs.anl.gov> - initial prototype\end{verbatim}
 
%/////////////////////////////////////////////////////////////
 
\mbox{}\hrulefill\ 
 

 \subsubsection{comm\_ - Return the communicator from the GlobalSegMap.}


 
  The function {\tt comm\_()} returns the fortran 90 integer handle
  corresponding to the communicator in the input {\tt GlobalSegMap}
  argument {\tt GSMap}.  This amounts to returning the value of 
  {\tt GSMap\%comm}.
 
\bigskip{\sf INTERFACE:}
\begin{verbatim} 
 function comm_(GSMap)
 
      implicit none
 
      type(GlobalSegMap),intent(in) :: GSMap
      integer :: comm_
 \end{verbatim}{\sf REVISION HISTORY:}
\begin{verbatim}  	29Sep98 - J.W. Larson <larson@mcs.anl.gov> - initial prototype\end{verbatim}
 
%/////////////////////////////////////////////////////////////
 
\mbox{}\hrulefill\ 
 

 \subsubsection{gsize\_ - Return the global vector size from the GlobalSegMap.}


 
  The function {\tt gsize\_()} takes the input {\tt GlobalSegMap} 
  arguement {\tt GSMap} and returns the global vector length stored
  in {\tt GlobalSegMap\%gsize}.
 
\bigskip{\sf INTERFACE:}
\begin{verbatim} 
 function gsize_(GSMap)
 
      implicit none
 
      type(GlobalSegMap),intent(in) :: GSMap
      integer :: gsize_
 \end{verbatim}{\sf REVISION HISTORY:}
\begin{verbatim}  	29Sep98 - J.W. Larson <larson@mcs.anl.gov> - initial prototype\end{verbatim}
 
%/////////////////////////////////////////////////////////////
 
\mbox{}\hrulefill\ 
 

 \subsubsection{lsize\_ - find the local storage size from the map}


 
 
\bigskip{\sf INTERFACE:}
\begin{verbatim} 
 function lsize_(GSMap)
 
      implicit none
 
      type(GlobalSegMap),intent(in) :: GSMap
      integer :: lsize_
 \end{verbatim}{\sf REVISION HISTORY:}
\begin{verbatim}  	29Sep98 - J.W. Larson <larson@mcs.anl.gov> - initial prototype\end{verbatim}
 
%/////////////////////////////////////////////////////////////
 
\mbox{}\hrulefill\ 
 

 \subsubsection{rank1\_ - rank which process owns a datum with given global }


  index.
 
  This routine assumes that there is one process that owns the datum with
  a given global index.  It should not be used when the input 
  {\tt GlobalSegMap} argument {\tt GSMap} has been built to incorporate
  halo points.
 
\bigskip{\sf INTERFACE:}
\begin{verbatim} 
    subroutine rank1_(GSMap, i_g, rank)
 
      implicit none
 
      type(GlobalSegMap), intent(in) :: GSMap     input GlobalSegMap
      integer,            intent(in) :: i_g	  a global index
      integer,           intent(out) :: rank      the pe on which this
                                                  element resides\end{verbatim}{\sf REVISION HISTORY:}
\begin{verbatim}  	29Sep98 - J.W. Larson <larson@mcs.anl.gov> - initial prototype\end{verbatim}
 
%/////////////////////////////////////////////////////////////
 
\mbox{}\hrulefill\ 
 

 \subsubsection{rankm\_ - rank which process owns a datum with given global }


  index.
 
  This routine assumes that there is one process that owns the datum with
  a given global index.  It should not be used when the input 
  {\tt GlobalSegMap} argument {\tt GSMap} has been built to incorporate
  halo points.  {\em Nota Bene}:  The output array {\tt rank} is allocated 
  in this routine and must be deallocated by the routine calling 
  {\tt rankm\_()}.  Failure to do so could result in a memory leak.
 
\bigskip{\sf INTERFACE:}
\begin{verbatim} 
    subroutine rankm_(GSMap, i_g, num_loc, rank)
 
      implicit none
 
      type(GlobalSegMap), intent(in) :: GSMap     input GlobalSegMap
      integer,            intent(in) :: i_g	  a global index
      integer,           intent(out) :: num_loc   the number of processes
                                                  which own element i_g
      integer, dimension(:), pointer :: rank      the process(es) on which 
                                                  element i_g resides\end{verbatim}{\sf REVISION HISTORY:}
\begin{verbatim}  	29Sep98 - J.W. Larson <larson@mcs.anl.gov> - initial prototype\end{verbatim}

%...............................................................

\section{Documentation of the AttrVectComms Module}
%                **** IMPORTANT NOTICE *****
% This LaTeX file has been automatically produced by ProTeX v. 1.0
% Any changes made to this file will likely be lost next time
% this file is regenerated from its Fortran source.
% Send questions to Arlindo da Silva, dasilva@gsfc.nasa.gov
 
\parskip        0pt
\parindent      0pt
\baselineskip  11pt
 
%--------------------- SHORT-HAND MACROS ----------------------
\def\bv{\begin{verbatim}}
\def\ev{\end{verbatim}}
\def\be{\begin{equation}}
\def\ee{\end{equation}}
\def\bea{\begin{eqnarray}}
\def\eea{\end{eqnarray}}
\def\bi{\begin{itemize}}
\def\ei{\end{itemize}}
\def\bn{\begin{enumerate}}
\def\en{\end{enumerate}}
\def\bd{\begin{description}}
\def\ed{\end{description}}
\def\({\left (}
\def\){\right )}
\def\[{\left [}
\def\]{\right ]}
\def\<{\left  \langle}
\def\>{\right \rangle}
\def\cI{{\cal I}}
\def\diag{\mathop{\rm diag}}
\def\tr{\mathop{\rm tr}}
%-------------------------------------------------------------
 
%/////////////////////////////////////////////////////////////

 \subsection{Module m\_AttrVectComms - Communications methods for the AttrVect}


 
 
  In this module, we define communications methods specific to the 
  {\tt AttrVect} class (see m\_AttrVect for more information about this
  class and its methods).
 
\bigskip{\sf INTERFACE:}
\begin{verbatim} module m_AttrVectComms\end{verbatim}{\em USES:}
\begin{verbatim}      use m_AttrVect   AttrVect class and its methods
 
 
      implicit none
 
      private	  except
 
      public :: gather		  gather all local vectors to the root
      public :: scatter		  scatter from the root to all PEs
      public :: bcast		  bcast from root to all PEs
 
    interface gather ; module procedure gather_ ; end interface
    interface scatter; module procedure scatter_; end interface
    interface bcast  ; module procedure bcast_  ; end interface
 \end{verbatim}{\sf REVISION HISTORY:}
\begin{verbatim}  	27Oct00 - J.W. Larson <larson@mcs.anl.gov> - relocated routines
                  from m_AttrVect to create this module.\end{verbatim}
 
%/////////////////////////////////////////////////////////////
 
\mbox{}\hrulefill\ 
 

 \subsubsection{gather\_ - gather a vector according to a given \_map\_}


 
 
\bigskip{\sf INTERFACE:}
\begin{verbatim} 
 subroutine gather_(iV,oV,Mp,root,comm,stat)\end{verbatim}{\em USES:}
\begin{verbatim}      use m_stdio
      use m_die
      use m_mpif90
      use m_GlobalMap, only : GlobalMap
      use m_GlobalMap, only : GlobalMap_lsize => lsize
      use m_GlobalMap, only : GlobalMap_gsize => gsize
      use m_AttrVect, only : AttrVect
      use m_AttrVect, only : AttrVect_init => init
      use m_AttrVect,  only : AttrVect_lsize => lsize
      use m_AttrVect, only : AttrVect_nIAttr => nIAttr
      use m_AttrVect, only : AttrVect_nRAttr => nRAttr
 
      implicit none
 
      type(AttrVect),intent(in)  :: iV
      type(AttrVect),intent(out) :: oV
      type(GlobalMap) ,intent(in)  :: Mp
      integer, intent(in) :: root
      integer, intent(in) :: comm
      integer, optional,intent(out) :: stat
 \end{verbatim}{\sf REVISION HISTORY:}
\begin{verbatim}  	15Apr98 - Jing Guo <guo@thunder> - initial prototype/prolog/code
  	27Oct00 - J.W. Larson <larson@mcs.anl.gov> - relocated from
                  m_AttrVect\end{verbatim}
 
%/////////////////////////////////////////////////////////////
 
\mbox{}\hrulefill\ 
 

 \subsubsection{scatter\_ - scatter a vecter according to a given \_map\_}


 
 
\bigskip{\sf INTERFACE:}
\begin{verbatim} 
 subroutine scatter_(iV,oV,Mp,root,comm,stat)\end{verbatim}{\em USES:}
\begin{verbatim}      use m_stdio
      use m_die
      use m_mpif90
      use m_GlobalMap, only : GlobalMap
      use m_GlobalMap, only : GlobalMap_lsize => lsize
      use m_GlobalMap, only : GlobalMap_gsize => gsize
      use m_AttrVect, only : AttrVect
      use m_AttrVect, only : AttrVect_init => init
      use m_AttrVect,  only : AttrVect_lsize => lsize
      use m_AttrVect, only : AttrVect_nIAttr => nIAttr
      use m_AttrVect, only : AttrVect_nRAttr => nRAttr
 
      implicit none
 
      type(AttrVect),intent(in)  :: iV
      type(AttrVect),intent(out) :: oV
      type(GlobalMap) ,intent(in)  :: Mp
      integer, intent(in) :: root
      integer, intent(in) :: comm
      integer, optional,intent(out) :: stat
 \end{verbatim}{\sf REVISION HISTORY:}
\begin{verbatim}  	21Apr98 - Jing Guo <guo@thunder> - initial prototype/prolog/code
  	27Oct00 - J.W. Larson <larson@mcs.anl.gov> - relocated from
                  m_AttrVect\end{verbatim}
 
%/////////////////////////////////////////////////////////////
 
\mbox{}\hrulefill\ 
 

 \subsubsection{bcast\_ - broadcast from the root to all PEs}


 
 
\bigskip{\sf INTERFACE:}
\begin{verbatim} 
 subroutine bcast_(aV,root,comm,stat)\end{verbatim}{\em USES:}
\begin{verbatim}      use m_die, only : die, perr
      use m_mpif90
      use m_String, only : String,bcast,char
      use m_String, only : String_bcast => bcast
      use m_List, only : List_get => get
      use m_AttrVect, only : AttrVect
      use m_AttrVect, only : AttrVect_init => init
      use m_AttrVect, only : AttrVect_lsize => lsize
      use m_AttrVect, only : AttrVect_nIAttr => nIAttr
      use m_AttrVect, only : AttrVect_nRAttr => nRAttr
 
      implicit none
 
      type(AttrVect) :: aV	  (IN) on the root, (OUT) elsewhere
      integer,intent(in) :: root
      integer,intent(in) :: comm
      integer,optional,intent(out) :: stat
 \end{verbatim}{\sf REVISION HISTORY:}
\begin{verbatim}  	27Apr98 - Jing Guo <guo@thunder> - initial prototype/prolog/code
  	27Oct00 - J.W. Larson <larson@mcs.anl.gov> - relocated from
                  m_AttrVect\end{verbatim}

%...............................................................

\part{Level 4 API's}
\section{Documentation of the Accumulator Module}
%                **** IMPORTANT NOTICE *****
% This LaTeX file has been automatically produced by ProTeX v. 1.0
% Any changes made to this file will likely be lost next time
% this file is regenerated from its Fortran source.
% Send questions to Arlindo da Silva, dasilva@gsfc.nasa.gov
 
\parskip        0pt
\parindent      0pt
\baselineskip  11pt
 
%--------------------- SHORT-HAND MACROS ----------------------
\def\bv{\begin{verbatim}}
\def\ev{\end{verbatim}}
\def\be{\begin{equation}}
\def\ee{\end{equation}}
\def\bea{\begin{eqnarray}}
\def\eea{\end{eqnarray}}
\def\bi{\begin{itemize}}
\def\ei{\end{itemize}}
\def\bn{\begin{enumerate}}
\def\en{\end{enumerate}}
\def\bd{\begin{description}}
\def\ed{\end{description}}
\def\({\left (}
\def\){\right )}
\def\[{\left [}
\def\]{\right ]}
\def\<{\left  \langle}
\def\>{\right \rangle}
\def\cI{{\cal I}}
\def\diag{\mathop{\rm diag}}
\def\tr{\mathop{\rm tr}}
%-------------------------------------------------------------
 
%/////////////////////////////////////////////////////////////

 \subsection{Module m\_Accumulator - a distributed accumulator for time-averaging.}


 
 
  An {\em accumulator} is a data class used for computing running sums 
  and/or time averages of {\tt AttrVect} class data.  The fortran 90 
  implementation of this concept is the {\tt Accumulator} class, 
  which---along with its basic methods---is defined in this module.
 
\bigskip{\sf INTERFACE:}
\begin{verbatim} 
 module m_Accumulator\end{verbatim}{\em USES:}
\begin{verbatim}      use m_List, only : List
      use m_AttrVect, only : AttrVect
 
      implicit none
 
      private	  except
 
  The class data structure
 
      public :: Accumulator     
 
  List of Methods for the Accumulator class
 
      public :: init		  creation method
      public :: clean		  destruction method
      public :: lsize		  local length of the data arrays
      public :: nIAttr		  number of integer fields
      public :: nRAttr		  number of real fields
      public :: indexIA		  index the integer fields
      public :: indexRA		  index the real fields
 
  Definition of the Accumulator class:
 
    type Accumulator
      integer :: num_steps        total number of accumulation steps
      integer :: steps_done       number of accumulation steps performed
      type(AttrVect) :: av        accumulated field storage
    end type Accumulator
 
  Definition of interfaces for the methods for the Accumulator:
 
    interface init   ; module procedure	&
 init_,	&
 initv_
    end interface
    interface clean  ; module procedure clean_  ; end interface
    interface lsize  ; module procedure lsize_  ; end interface
    interface nIAttr ; module procedure nIAttr_ ; end interface
    interface nRAttr ; module procedure nRAttr_ ; end interface
    interface indexIA; module procedure indexIA_; end interface
    interface indexRA; module procedure indexRA_; end interface
 \end{verbatim}{\sf REVISION HISTORY:}
\begin{verbatim}  	 7Sep00 - Jay Larson <larson@mcs.anl.gov> - initial prototype\end{verbatim}
 
%/////////////////////////////////////////////////////////////
 
\mbox{}\hrulefill\ 
 

 \subsubsection{init\_ - initialize with given iList, rList, length, }


  num\_steps, and steps\_done.
 
 
\bigskip{\sf INTERFACE:}
\begin{verbatim} 
 subroutine init_(aC,iList,rList,lsize,num_steps,steps_done)\end{verbatim}{\em USES:}
\begin{verbatim}      use m_List, only : List_init=>init
      use m_List, only : List_nitem=>nitem
      use m_AttrVect, only : AttrVect_init => init
      use m_die
 
      implicit none
 
      type(Accumulator),intent(out)        :: aC
      character(len=*),optional,intent(in) :: iList
      character(len=*),optional,intent(in) :: rList
      integer,         optional,intent(in) :: lsize
      integer,         intent(in)          :: num_steps
      integer,         optional,intent(in) :: steps_done
 \end{verbatim}{\sf REVISION HISTORY:}
\begin{verbatim}  	11Sep00 - Jay Larson <larson@mcs.anl.gov> - initial prototype\end{verbatim}
 
%/////////////////////////////////////////////////////////////
 
\mbox{}\hrulefill\ 
 

 \subsubsection{initv\_-Initialize an accumulator using another Accumulator.}


 
 
\bigskip{\sf INTERFACE:}
\begin{verbatim} 
 subroutine initv_(aC,bC,lsize,num_steps,steps_done)\end{verbatim}{\em USES:}
\begin{verbatim}      use m_String, only : String
      use m_String, only : String_char => char
      use m_List,   only : List_get => get
 
      implicit none
 
      type(Accumulator),    intent(out) :: aC
      type(Accumulator),    intent(in)  :: bC
      integer,           intent(in)  :: lsize
      integer,           intent(in)  :: num_steps
      integer, optional, intent(in)  :: steps_done
 \end{verbatim}{\sf REVISION HISTORY:}
\begin{verbatim}  	11Sep00 - Jay Larson <larson@mcs.anl.gov> - initial prototype\end{verbatim}
 
%/////////////////////////////////////////////////////////////
 
\mbox{}\hrulefill\ 
 

 \subsubsection{clean\_ - Destruction method for the Accumulator.}


 
 
\bigskip{\sf INTERFACE:}
\begin{verbatim} 
 subroutine clean_(aC)\end{verbatim}{\em USES:}
\begin{verbatim}      use m_mall
      use m_stdio
      use m_die
      use m_AttrVect, only : AttrVect_clean => clean
 
      implicit none
 
      type(Accumulator),intent(inout) :: aC
 \end{verbatim}{\sf REVISION HISTORY:}
\begin{verbatim}  	11Sep00 - Jay Larson <larson@mcs.anl.gov> - initial prototype\end{verbatim}
 
%/////////////////////////////////////////////////////////////
 
\mbox{}\hrulefill\ 
 

 \subsubsection{lsize\_ - local size of data storage in the Accumulator.}


 
 
\bigskip{\sf INTERFACE:}
\begin{verbatim} 
 function lsize_(aC)\end{verbatim}{\em USES:}
\begin{verbatim}      use m_AttrVect, only : AttrVect_lsize => lsize
 
      implicit none
 
      type(Accumulator), intent(in) :: aC
      integer :: lsize_
 \end{verbatim}{\sf REVISION HISTORY:}
\begin{verbatim}  	12Sep00 - Jay Larson <larson@mcs.anl.gov> - initial prototype\end{verbatim}
 
%/////////////////////////////////////////////////////////////
 
\mbox{}\hrulefill\ 
 

 \subsubsection{nIAttr\_ - number of INTEGER fields stored in the Accumulator.}


 
 
\bigskip{\sf INTERFACE:}
\begin{verbatim} 
    function nIAttr_(aC)\end{verbatim}{\em USES:}
\begin{verbatim}      use m_AttrVect, only : AttrVect_nIAttr => nIAttr
 
      implicit none
 
      type(Accumulator),intent(in) :: aC
      integer :: nIAttr_
 \end{verbatim}{\sf REVISION HISTORY:}
\begin{verbatim}  	12Sep00 - Jay Larson <larson@mcs.anl.gov> - initial prototype\end{verbatim}
 
%/////////////////////////////////////////////////////////////
 
\mbox{}\hrulefill\ 
 

 \subsubsection{nRAttr\_ - number of REAL fields stored in the Accumulator.}


 
 
\bigskip{\sf INTERFACE:}
\begin{verbatim} 
 function nRAttr_(aC)\end{verbatim}{\em USES:}
\begin{verbatim}      use m_AttrVect, only : AttrVect_nRAttr => nRAttr
 
      implicit none
 
      type(Accumulator),intent(in) :: aC
      integer :: nRAttr_
 \end{verbatim}{\sf REVISION HISTORY:}
\begin{verbatim}  	12Sep00 - Jay Larson <larson@mcs.anl.gov> - initial prototype\end{verbatim}
 
%/////////////////////////////////////////////////////////////
 
\mbox{}\hrulefill\ 
 

 \subsubsection{getIList\_ - get an item from the integer attribute list }


  in the accumulator's data storage area (i.e. its AttrVect component).
 
 
\bigskip{\sf INTERFACE:}
\begin{verbatim} 
    subroutine getIList_(item,ith,aC)\end{verbatim}{\em USES:}
\begin{verbatim}      use m_AttrVect, only : AttrVect_getIList => getIList
      use m_String,   only : String
 
      implicit none
      type(String),intent(out)     :: item
      integer,     intent(in)      :: ith
      type(Accumulator),intent(in) :: aC
 \end{verbatim}{\sf REVISION HISTORY:}
\begin{verbatim}  	12Sep00 - Jay Larson <larson@mcs.anl.gov> - initial prototype\end{verbatim}
 
%/////////////////////////////////////////////////////////////
 
\mbox{}\hrulefill\ 
 

 \subsubsection{getRList\_ - get an item from real attribute list in the}


  accumulator's data storage space (i.e. its AttrVect component).
 
 
\bigskip{\sf INTERFACE:}
\begin{verbatim} 
    subroutine getRList_(item,ith,aC)\end{verbatim}{\em USES:}
\begin{verbatim}      use m_AttrVect, only : AttrVect_getRList => getRList
      use m_String,   only : String
 
      implicit none
      type(String),     intent(out) :: item
      integer,          intent(in)  :: ith
      type(Accumulator),intent(in)  :: aC
 \end{verbatim}{\sf REVISION HISTORY:}
\begin{verbatim}  	12Sep00 - Jay Larson <larson@mcs.anl.gov> - initial prototype\end{verbatim}
 
%/////////////////////////////////////////////////////////////
 
\mbox{}\hrulefill\ 
 

 \subsubsection{indexIA\_ - index the Accumulator's integer attribute List.}


 
 
\bigskip{\sf INTERFACE:}
\begin{verbatim} 
 function indexIA_(aC,item,perrWith,dieWith)\end{verbatim}{\em USES:}
\begin{verbatim}      use m_AttrVect, only : AttrVect_indexIA => indexIA
      use m_die,  only : die
      use m_stdio,only : stderr
 
      implicit none
 
     type(Accumulator), intent(in) :: aC
      character(len=*),intent(in) :: item
      character(len=*),optional,intent(in) :: perrWith
      character(len=*),optional,intent(in) :: dieWith
      integer :: indexIA_
 \end{verbatim}{\sf REVISION HISTORY:}
\begin{verbatim}  	14Sep00 - Jay Larson <larson@mcs.anl.gov> - initial prototype\end{verbatim}
 
%/////////////////////////////////////////////////////////////
 
\mbox{}\hrulefill\ 
 

 \subsubsection{indexRA\_ - index the Accumulator's real attribute list.}


 
 
\bigskip{\sf INTERFACE:}
\begin{verbatim} 
 function indexRA_(aC,item,perrWith,dieWith)\end{verbatim}{\em USES:}
\begin{verbatim}      use m_AttrVect, only : AttrVect_indexRA => indexRA
      use m_die,  only : die
      use m_stdio,only : stderr
 
      implicit none
 
      type(Accumulator), intent(in) :: aC
      character(len=*),intent(in) :: item
      character(len=*),optional,intent(in) :: perrWith
      character(len=*),optional,intent(in) :: dieWith
      integer :: indexRA_
 \end{verbatim}{\sf REVISION HISTORY:}
\begin{verbatim}  	14Sep00 - Jay Larson <larson@mcs.anl.gov> - initial prototype\end{verbatim}

%...............................................................

\section{Documentation of the AccumulatorComms Module}
%                **** IMPORTANT NOTICE *****
% This LaTeX file has been automatically produced by ProTeX v. 1.0
% Any changes made to this file will likely be lost next time
% this file is regenerated from its Fortran source.
% Send questions to Arlindo da Silva, dasilva@gsfc.nasa.gov
 
\parskip        0pt
\parindent      0pt
\baselineskip  11pt
 
%--------------------- SHORT-HAND MACROS ----------------------
\def\bv{\begin{verbatim}}
\def\ev{\end{verbatim}}
\def\be{\begin{equation}}
\def\ee{\end{equation}}
\def\bea{\begin{eqnarray}}
\def\eea{\end{eqnarray}}
\def\bi{\begin{itemize}}
\def\ei{\end{itemize}}
\def\bn{\begin{enumerate}}
\def\en{\end{enumerate}}
\def\bd{\begin{description}}
\def\ed{\end{description}}
\def\({\left (}
\def\){\right )}
\def\[{\left [}
\def\]{\right ]}
\def\<{\left  \langle}
\def\>{\right \rangle}
\def\cI{{\cal I}}
\def\diag{\mathop{\rm diag}}
\def\tr{\mathop{\rm tr}}
%-------------------------------------------------------------
 
%/////////////////////////////////////////////////////////////

 \subsection{Module m\_AccumulatorComms - Communication methods for the }


           the Accumulator class.
 
 
  An {\em accumulator} is a data class used for computing running sums 
  and/or time averages of {\tt AttrVect} class data (see 
  {\tt m\_Accumulator} for details).  This module defines the 
  communications methods for the accumulator, employing both the 
  {\tt GlobalMap} and {\tt GlobalSegMap} decomposition descriptors.
 
\bigskip{\sf INTERFACE:}
\begin{verbatim} 
 module m_AccumulatorComms\end{verbatim}{\em USES:}
\begin{verbatim}      use m_Accumulator, only : Accumulator
      use m_GlobalMap,   only : GlobalMap
 
      implicit none
 
      private	  except
 
  List of communications Methods for the Accumulator class
 
      public :: gather		  gather all local vectors to the root
      public :: scatter		  scatter from the root to all PEs
      public :: bcast		  bcast from root to all PEs
 
  Definition of interfaces for the communication methods for 
  the Accumulator:
 
    interface gather ; module procedure gather_ ; end interface
    interface scatter; module procedure scatter_; end interface
    interface bcast  ; module procedure bcast_  ; end interface
 \end{verbatim}{\sf REVISION HISTORY:}
\begin{verbatim}  	31Oct00 - Jay Larson <larson@mcs.anl.gov> - initial prototype--
                  These routines were separated from the module 
                  {\tt m\_Accumulator}\end{verbatim}
 
%/////////////////////////////////////////////////////////////
 
\mbox{}\hrulefill\ 
 

 \subsubsection{gather\_ - gather a vector according to a given GlobalMap}


 
 
\bigskip{\sf INTERFACE:}
\begin{verbatim} 
 subroutine gather_(iC,oC,Mp,root,comm,stat)\end{verbatim}{\em USES:}
\begin{verbatim}      use m_stdio
      use m_die
      use m_GlobalMap, only : GlobalMap
      use m_GlobalMap, only : GlobalMap_lsize => lsize
      use m_GlobalMap, only : GlobalMap_gsize => gsize
      use m_AttrVect,  only : AttrVect_lsize => lsize
      use m_Accumulator,  only : Accumulator_lsize => lsize
      use m_AttrVectComms,  only : AttrVect_gather => gather
      use m_mpif90
 
      implicit none
      type(Accumulator),intent(in)  :: iC
      type(Accumulator),intent(out) :: oC
      type(GlobalMap) ,intent(in)   :: Mp
      integer, intent(in) :: root
      integer, intent(in) :: comm
      integer, optional,intent(out) :: stat
 \end{verbatim}{\sf REVISION HISTORY:}
\begin{verbatim}  	13Sep00 - Jay Larson <larson@mcs.anl.gov> - initial prototype
  	31Oct00 - Jay Larson <larson@mcs.anl.gov> - relocated to the
                  module m_AccumulatorComms\end{verbatim}
 
%/////////////////////////////////////////////////////////////
 
\mbox{}\hrulefill\ 
 

 \subsubsection{scatter\_ - scatter an Accumulator using a GlobalMap.}


 
 
\bigskip{\sf INTERFACE:}
\begin{verbatim} 
 subroutine scatter_(iC,oC,Mp,root,comm,stat)\end{verbatim}{\em USES:}
\begin{verbatim}      use m_stdio
      use m_die
      use m_GlobalMap, only : GlobalMap_lsize => lsize
      use m_GlobalMap, only : GlobalMap_gsize => gsize
      use m_Accumulator, only : Accumulator_lsize => lsize
      use m_mpif90
      use m_AttrVectComms, only : AttrVect_scatter => scatter
 
      implicit none
      type(Accumulator),intent(in)  :: iC
      type(Accumulator),intent(out) :: oC
      type(GlobalMap) ,intent(in)  :: Mp
      integer, intent(in) :: root
      integer, intent(in) :: comm
      integer, optional,intent(out) :: stat
 \end{verbatim}{\sf REVISION HISTORY:}
\begin{verbatim}  	14Sep00 - Jay Larson <larson@mcs.anl.gov> - initial prototype
  	31Oct00 - Jay Larson <larson@mcs.anl.gov> - moved from the module
                  m_Accumulator to m_AccumulatorComms\end{verbatim}
 
%/////////////////////////////////////////////////////////////
 
\mbox{}\hrulefill\ 
 

 \subsubsection{bcast\_ - broadcast an Accumulator from the root to all PEs.}


 
 
\bigskip{\sf INTERFACE:}
\begin{verbatim} 
 subroutine bcast_(aC, root, comm, stat)\end{verbatim}{\em USES:}
\begin{verbatim}      use m_die, only : die, perr
      use m_mpif90
      use m_AttrVectComms, only : AttrVect_bcast => bcast
 
      implicit none
 
      type(Accumulator)  :: aC	  (IN) on the root, (OUT) elsewhere
      integer,intent(in) :: root
      integer,intent(in) :: comm
      integer,optional,intent(out) :: stat
 \end{verbatim}{\sf REVISION HISTORY:}
\begin{verbatim}  	14Sep00 - Jay Larson <larson@mcs.anl.gov> - initial prototype
  	31Oct00 - Jay Larson <larson@mcs.anl.gov> - moved from the module
                  m_Accumulator to m_AccumulatorComms\end{verbatim}

%...............................................................

\section{Documentaiton of the SparseMatrix Module}
%                **** IMPORTANT NOTICE *****
% This LaTeX file has been automatically produced by ProTeX v. 1.0
% Any changes made to this file will likely be lost next time
% this file is regenerated from its Fortran source.
% Send questions to Arlindo da Silva, dasilva@gsfc.nasa.gov
 
\parskip        0pt
\parindent      0pt
\baselineskip  11pt
 
%--------------------- SHORT-HAND MACROS ----------------------
\def\bv{\begin{verbatim}}
\def\ev{\end{verbatim}}
\def\be{\begin{equation}}
\def\ee{\end{equation}}
\def\bea{\begin{eqnarray}}
\def\eea{\end{eqnarray}}
\def\bi{\begin{itemize}}
\def\ei{\end{itemize}}
\def\bn{\begin{enumerate}}
\def\en{\end{enumerate}}
\def\bd{\begin{description}}
\def\ed{\end{description}}
\def\({\left (}
\def\){\right )}
\def\[{\left [}
\def\]{\right ]}
\def\<{\left  \langle}
\def\>{\right \rangle}
\def\cI{{\cal I}}
\def\diag{\mathop{\rm diag}}
\def\tr{\mathop{\rm tr}}
%-------------------------------------------------------------
 
%/////////////////////////////////////////////////////////////

 \subsection{Module m\_SparseMatrix -- Sparse Matrix class and methods.}


 
  The SparseMatrix data type is a special case of the  AttrVect data 
  type (see m\_AttrVect for details).  This data type has two storage 
  arrays, one for integer attributes (SparseMatrix\%iAttr) to hold row
  and column data, and one for real attributes (symMatx\%rAttr) which 
  holds the matrix element for that row and column.
 
  The set of attributes in each storage array are defined by a List; 
  SparseMatrix\%iList for integer attributes, and SparseMatrix\%rList 
  for real attributes.  The integer and real attribute tags are defined
  below:
 
  SparseMatrix\%iList components:
     row : row index
     col : column index
 
  SparseMatrix\%rList components:
     weight : matrix element
 
\bigskip{\sf INTERFACE:}
\begin{verbatim} 
 module m_SparseMatrix\end{verbatim}{\em USES:}
\begin{verbatim}      use m_AttrVect, only : SparseMatrix => AttrVect
      use m_AttrVect, only : AttrVect_init => init
      use m_AttrVectComms, only : gather
      use m_AttrVectComms, only : scatter
      use m_AttrVectComms, only : bcast
 
      implicit none
 
      private     except
 
      public :: SparseMatrix      The class data structure
      public :: init              Create a SparseMatrix
      public :: clean             Destroy a SparseMatrix
      public :: gather            Gather a SparseMatrix
      public :: scatter           Scatter a SparseMatrix
      public :: bcast             Broadcast a SparseMatrix
 
    interface init  ; module procedure init_  ; end interface
    interface clean ; module procedure clean_ ; end interface
 \end{verbatim}{\sf REVISION HISTORY:}
\begin{verbatim}        19Sep00 - J.W. Larson <larson@mcs.anl.gov> - initial prototype\end{verbatim}
 
%/////////////////////////////////////////////////////////////
 
\mbox{}\hrulefill\ 
 

 \subsubsection{init\_ - initialize a SparseMatrix}


 
 
\bigskip{\sf INTERFACE:}
\begin{verbatim} 
 subroutine init_(sMat, lsize )\end{verbatim}{\em USES:}
\begin{verbatim}      use m_AttrVect, only : AttrVect_init => init
      use m_die
 
      implicit none
 
      type(SparseMatrix), intent(out)        :: sMat
      integer,         optional,intent(in)   :: lsize
 \end{verbatim}{\sf REVISION HISTORY:}
\begin{verbatim}        19Sep00 - Jay Larson <larson@mcs.anl.gov> - initial prototype\end{verbatim}
 
%/////////////////////////////////////////////////////////////
 
\mbox{}\hrulefill\ 
 

 \subsubsection{clean\_ - Destroy a SparseMatrix.}


 
 
\bigskip{\sf INTERFACE:}
\begin{verbatim} 
    subroutine clean_(sMat)\end{verbatim}{\em USES:}
\begin{verbatim}      use m_AttrVect,only : AttrVect_clean => clean
 
      implicit none
 
      type(SparseMatrix), intent(inout) :: sMat
 \end{verbatim}{\sf REVISION HISTORY:}
\begin{verbatim}        19Sep00 - J.W. Larson <larson@mcs.anl.gov> - initial prototype\end{verbatim}

%...............................................................


%\addcontentsline{toc}{part}{References}

%\bibliographystyle{apalike}   % for BibTeX - uses [Name, year] method??
 
%\bibliography{coupler}
\end{document}
